\chapter{Broncodes kernels}
%\section{Kernels om $f(\T, \mUUU)$ te berekenen}
\label{app:A}
In deze appendix kan u de volledige broncodes vinden voor alle besproken kernels uit hoofdstuk \ref{h:kernels}. Ze staan in de volgorde waarin ze besproken worden in de tekst. We raden de lezer aan om de broncodes stap voor stap te bestuderen. Zo zal de lezer sneller inzicht krijgen in de structuur van de geavanceerde kernels. Omdat niet alle kernels op \'e\'en blad passen gaan we de ondertitels boven de code zetten.

\newpage
\lstinputlisting[caption={Broncode van de eerste kernel.}, label={codeFloat}]{opencl/float.cl}

\newpage
\lstinputlisting[caption={Broncode van de eerste kernel met 64 work-items.}, label={codeFloat64}]{opencl/float64.cl}

\newpage
\lstinputlisting[caption={Broncode van float4x4x4.}, label={codeFloat4x4x4}]{opencl/float4x4x4.cl}

\newpage
\lstinputlisting[caption={Broncode van float8x8x8.}, label={codeFloat8x8x8}]{opencl/float8x8x8.cl}

\newpage
\lstinputlisting[caption={Broncode van float16x16x16.}, label={codeFloat16x16x16}]{opencl/float16x16x16.cl}

\newpage
\lstCode{float8x8x8R}

\newpage
\lstCode{float16x16x16R}

\newpage
\lstCode{float8x8x8Mapper}

\newpage
\lstCode{float16x16x16Mapper}

\newpage
\lstCode{float16x16x16I}

\newpage
\lstCode{float16x16x16MapperI}

\newpage
\lstCode{double16x16x16FG}
%%% Local Variables: 
%%% mode: latex
%%% TeX-master: "masterproef"
%%% End: 
